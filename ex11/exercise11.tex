\documentclass[a4paper,12pt]{article}
\usepackage[ngerman]{babel}
\usepackage[utf8]{inputenc}
\usepackage{tikz}
\usetikzlibrary{automata, arrows, fit}
\usepackage{marginnote}
\usepackage{lipsum}
\usepackage{graphicx}
\usepackage{epstopdf}
\usepackage{amsmath, tabu}
\usepackage{amsfonts}
\usepackage{geometry}
\usepackage{mathtools}
\geometry{
	headsep=0.5cm,
	headheight=2.5cm,
	marginparwidth=2cm,
	textheight=22cm,
}
\setlength{\headheight}{120pt}
\usepackage{fancyhdr}
\setlength\parindent{0pt}
\renewcommand{\familydefault}{\sfdefault}
\renewcommand{\arraystretch}{1.2}
\allowdisplaybreaks
\graphicspath{{../figures/}}


\pagestyle{fancy}
\fancyhf{}

\lhead{
	\textit{Humboldt-Universität zu Berlin, Institut der Informatik}\\
	\bigskip
	\textbf{\Large{Werkzeuge der empirischen Forschung}}\\
	\bigskip
	Abgabe: 01.07.2019\\
	Blatt 11\\[-1em]\\
	Pohl, Oliver \\	 577878\\  pohloliq\\[-1em]
}

\newcommand*{\QED}{\hfill\ensuremath{q.e.d.}}
\newcommand{\ex}[1]{\newpage\subsubsection*{Aufgabe #1.}}
\newcommand*{\simpleTable}[2]{
	\begin{tabular}{@{} #1 @{}}
		\toprule
		#2
		\bottomrule
	\end{tabular}
}
\DeclareMathOperator{\sig}{sig}

\newcommand\addvmargin[1]{
  \node[fit=(current bounding box),inner ysep=#1,inner xsep=0]{};
}

\begin{document}
	
	


	%%%%%%%%%%%%%%%%%%%% Aufgabe 31 %%%%%%%%%%%%%%%%%%%%%%%%%%%%%%%%%%%%%%%%%%%%%%%%%
\ex{31}

$H_0$: Merkmale sind unabhängig \\♣
$H_1$: $\neg$ $H_0$\\

Beobachtete Werte:\\

\begin{tabular}{ | l | l | l | p{5cm} |}
	\hline
	$n_0$ & ländlich & städtisch &   \\ \hline
	Küstentiefland & 6 & 7 & 13\\ \hline
	Inneres Hochland & 9 & 8 & 17 \\ \hline
	 & 15 & 15 & 30\\ \hline

\end{tabular}

Erwartete Werte:\\

\begin{tabular}{ | l | l | l | p{5cm} |}
	\hline
	$n_e$ & ländlich & städtisch &   \\ \hline
	Küstentiefland & 6.5 & 6.5 & 13\\ \hline
	Inneres Hochland & 8.5 & 8.5 & 17 \\ \hline
	& 15 & 15 & 30\\ \hline
	
\end{tabular}


Freiheitsgrade:  df=1 , Vergleichswert = 3.84\\
Chiquadrat-Teststatistik berechnen:\\

	\begin{align*}
\chi^2 
&= \frac{(6-6.5)^2}{6.5}+\frac{(7-6.5)^2}{6.5}+\frac{(9-8.5)^2}{8.5}+\frac{(8-8.5)^2}{8.5} \\
&= 0.0384 + 0.0384 + 0.0294 + 0.0294\\
&= 	0.1356\\
\end{align*}

Da 0.1356 $<$ 3.84 wird die Nullhypothese angenommen . Die beiden Merkmale sind unabhängig mit einer Irrtumswahrscheinlichkeit von 0.05.

	
	%%%%%%%%%%%%%%%%%%%% Aufgabe 32a %%%%%%%%%%%%%%%%%%%%%%%%%%%%%%%%%%%%%%%%%%%%%%%%%
\ex{32a}


Folgende fiktive Beobachtungen sind gegeben:\\\\

\begin{tabular}{ | l | l | l | l | l | p{0.5cm} |}
	\hline
	$X:$ & 1 & 1 & 1 & 1 & 1 \\ \hline
	$Y:$ & 1 & 1 & 1 & 1 & 2 \\ \hline
\end{tabular}

Vergleich anhand eines Wilcoxon-Tests:\\

Zusammenfassen der Beobachtungen zu einer Stichprobe: \\

$[1,1,1,1,1,1,1,1,1,2]$

Rangzahlen bilden: \\

$[5,5,5,5,5,5,5,5,5,10]$


Berechnung der Summe der Ränge für Stichprobe X:
\begin{align*}
S_1 
&= \sum_{j=1}^{5} R_{1j} \\
&= \sum_{j=1}^{5} R_{1j} = 5+5+5+5+5 \\
&= \sum_{j=1}^{5} R_{1j} = 25	\\
\end{align*}

Berechnung der Erwartungswerte(unter H0):

\begin{align*}
E(S_1)
&= \frac{5\cdot(5+5+1)}{2} \\
&= 27.5 \\
\end{align*}


Berechnung der Varianzen:

\begin{align*}
Var(S_1)
&= \frac{5\cdot 5 \cdot (5+5+1)}{12} \\
&=  22.91667 \\
\end{align*}


Berechnung der Teststatistik des Wilcoxon-Test:

\begin{align*}
Z
&= \frac{S-E(S)}{\sqrt{varS}} \\
&= \frac{25-27.5}{\sqrt{22.92}}  \\
&=  -0.52 \\
\end{align*}

Berechnung des p-Wertes:

\begin{align*}
p
&= 2 \cdot pnorm(-0.52) \\
&= 2 \cdot 0.302  \\
&= 0.604 \\
\end{align*}

Da der p-Wert = 0.604 $>$ 0.05 nehmen wir die Nullhypothese an.

Folgende fiktive Beobachtungen sind gegeben:\\

\begin{tabular}{ | l | l | l | l | l | p{0.5cm} |}
	\hline
	$X:$ & 1 & 1 & 1 & 1 & 1 \\ \hline
	$Y:$ & 1 & 1 & 1 & 1 & 5 \\ \hline
\end{tabular}

Vergleich anhand eines Wilcoxon-Tests:\\

Zusammenfassen der Beobachtungen zu einer Stichprobe: \\

$[1,1,1,1,1,1,1,1,1,5]$

Rangzahlen bilden: \\

$[5,5,5,5,5,5,5,5,5,10]$


Berechnung der Summe der Ränge für Stichprobe Y:
\begin{align*}
S_1 
&= \sum_{j=1}^{5} R_{1j} \\
&= \sum_{j=1}^{5} R_{1j} = 5+5+5+5+10 \\
&= \sum_{j=1}^{5} R_{1j} = 30	\\
\end{align*}

Berechnung der Erwartungswerte(unter H0):

\begin{align*}
E(S_1)
&= \frac{5\cdot(5+5+1)}{2} \\
&= 27.5 \\
\end{align*}


Berechnung der Varianzen:

\begin{align*}
Var(S_1)
&= \frac{5\cdot 5 \cdot (5+5+1)}{12} \\
&=  22.91667 \\
\end{align*}


Berechnung der Teststatistik des Wilcoxon-Test:

\begin{align*}
Z
&= \frac{S-E(S)}{\sqrt{varS}} \\
&= \frac{30-27.5}{\sqrt{22.92}}  \\
&=  0.52 \\
\end{align*}

Berechnung des p-Wertes:

\begin{align*}
p
&= 2 \cdot pnorm(-0.52) \\
&= 2 \cdot 0.302  \\
&= 0.604 \\
\end{align*}

Da der p-Wert = 0.604 $>$ 0.05 nehmen wir die Nullhypothese an.

\\\\
Folgende fiktive Beobachtungen sind gegeben:\\

\begin{tabular}{ | l | l | l | l | l | p{0.5cm} |}
	\hline
	$X:$ & 1 & 1 & 1 & 1 & 1 \\ \hline
	$Y:$ & 1 & 1 & 1 & 1 & 10 \\ \hline
\end{tabular}

Vergleich anhand eines Wilcoxon-Tests:\\

Zusammenfassen der Beobachtungen zu einer Stichprobe: \\

$[1,1,1,1,1,1,1,1,1,5]$

Rangzahlen bilden: \\

$[5,5,5,5,5,5,5,5,5,10]$


Berechnung der Summe der Ränge für Stichprobe Y:
\begin{align*}
S_1 
&= \sum_{j=1}^{5} R_{1j} \\
&= \sum_{j=1}^{5} R_{1j} = 5+5+5+5+10 \\
&= \sum_{j=1}^{5} R_{1j} = 30	\\
\end{align*}

Berechnung der Erwartungswerte(unter H0):

\begin{align*}
E(S_1)
&= \frac{5\cdot(5+5+1)}{2} \\
&= 27.5 \\
\end{align*}


Berechnung der Varianzen:

\begin{align*}
Var(S_1)
&= \frac{5\cdot 5 \cdot (5+5+1)}{12} \\
&=  22.91667 \\
\end{align*}


Berechnung der Teststatistik des Wilcoxon-Test:

\begin{align*}
Z
&= \frac{S-E(S)}{\sqrt{varS}} \\
&= \frac{30-27.5}{\sqrt{22.92}}  \\
&=  0.52 \\
\end{align*}

Berechnung des p-Wertes:

\begin{align*}
p
&= 2 \cdot pnorm(-0.52) \\
&= 2 \cdot 0.302  \\
&= 0.604 \\
\end{align*}

Da der p-Wert = 0.604 $>$ 0.05 nehmen wir die Nullhypothese an.\\\\

Interpretation der Ergebnisse:\\

Egal wie groß der Ausreißer von der Stichprobe Y ist, der p-Wert ist immer derselbe.\\
Man kann an diesem Beispiel die Robustheit des Wilcoxon-Test sehen.



	%%%%%%%%%%%%%%%%%%%% Aufgabe 32b %%%%%%%%%%%%%%%%%%%%%%%%%%%%%%%%%%%%%%%%%%%%%%%%%
\ex{32b}


Folgende fiktive Beobachtungen sind gegeben:\\

\begin{tabular}{ | l | l | l | p{0.5cm} |}
	\hline
	$X:$ & 1 & 2 & 3  \\ \hline
	$Y:$ & 0 & 0 & 0  \\ \hline
\end{tabular}

Vergleich anhand eines Wilcoxon-Tests:\\

Zusammenfassen der Beobachtungen zu einer Stichprobe: \\

$[0,0,0,1,2,3]$

Rangzahlen bilden: \\

$[2,2,2,4,5,6]$


Berechnung der Summe der Ränge für Stichprobe X:
\begin{align*}
S_1 
&= \sum_{j=1}^{3} R_{1j} \\
&= \sum_{j=1}^{3} R_{1j} = 4+5+6 \\
&= \sum_{j=1}^{3} R_{1j} = 15	\\
\end{align*}

Berechnung der Erwartungswerte(unter H0):

\begin{align*}
E(S_1)
&= \frac{3\cdot(3+3+1)}{2} \\
&= 10.5 \\
\end{align*}


Berechnung der Varianzen:

\begin{align*}
Var(S_1)
&= \frac{3\cdot 3 \cdot (3+3+1)}{12} \\
&=  5.25 \\
\end{align*}


Berechnung der Teststatistik des Wilcoxon-Test:

\begin{align*}
Z
&= \frac{S-E(S)}{\sqrt{varS}} \\
&= \frac{15-10.5}{\sqrt{5.25}}  \\
&= 1.97 \\
\end{align*}

Berechnung des p-Wertes:

\begin{align*}
p
&= 2 \cdot pnorm(1.97) \\
&= 2 \cdot  0.02 \\
&= 0.04 \\
\end{align*}

Da der p-Wert = 0.04 $<$ 0.05 lehnen wir die Nullhypothese ab.\\\\


Folgende fiktive Beobachtungen sind gegeben:\\

\begin{tabular}{ | l | l | l | l | p{0.5cm} |}
	\hline
	$X:$ & 1 & 2 & 3 & 10 \\ \hline
	$Y:$ & 0 & 0 & 0  \\ \hline
\end{tabular}

Vergleich anhand eines Wilcoxon-Tests:\\

Zusammenfassen der Beobachtungen zu einer Stichprobe: \\

$[0,0,0,1,2,3,10]$

Rangzahlen bilden: \\

$[2,2,2,4,5,6,7]$


Berechnung der Summe der Ränge für Stichprobe Y:
\begin{align*}
S_1 
&= \sum_{j=1}^{3} R_{1j} \\
&= \sum_{j=1}^{3} R_{1j} = 2+2+2 \\
&= \sum_{j=1}^{3} R_{1j} = 6	\\
\end{align*}

Berechnung der Erwartungswerte(unter H0):

\begin{align*}
E(S_1)
&= \frac{3\cdot(3+4+1)}{2} \\
&= 12 \\
\end{align*}


Berechnung der Varianzen:

\begin{align*}
Var(S_1)
&= \frac{3\cdot 4 \cdot (3+4+1)}{12} \\
&=  8 \\
\end{align*}


Berechnung der Teststatistik des Wilcoxon-Test:

\begin{align*}
Z
&= \frac{S-E(S)}{\sqrt{varS}} \\
&= \frac{6-12}{\sqrt{8}}  \\
&= -2.12 \\
\end{align*}

Berechnung des p-Wertes:

\begin{align*}
p
&= 2 \cdot pnorm(-2.12) \\
&= 2 \cdot 0.017  \\
&= 0.034 \\
\end{align*}

Da der p-Wert = 0.034 $<$ 0.05 lehnen wir die Nullhypothese ab.\\\\


\end{document}
