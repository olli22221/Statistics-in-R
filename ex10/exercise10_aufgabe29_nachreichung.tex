\documentclass[a4paper,12pt]{article}
\usepackage[ngerman]{babel}
\usepackage[utf8]{inputenc}
\usepackage{tikz}
\usetikzlibrary{automata, arrows, fit}
\usepackage{marginnote}
\usepackage{lipsum}
\usepackage{graphicx}
\usepackage{epstopdf}
\usepackage{amsmath, tabu}
\usepackage{amsfonts}
\usepackage{geometry}
\usepackage{mathtools}
\geometry{
	headsep=0.5cm,
	headheight=2.5cm,
	marginparwidth=2cm,
	textheight=22cm,
}
\setlength{\headheight}{120pt}
\usepackage{fancyhdr}
\setlength\parindent{0pt}
\renewcommand{\familydefault}{\sfdefault}
\renewcommand{\arraystretch}{1.2}
\allowdisplaybreaks
\graphicspath{{../figures/}}


\pagestyle{fancy}
\fancyhf{}

\lhead{
	\textit{Humboldt-Universität zu Berlin, Institut der Informatik}\\
	\bigskip
	\textbf{\Large{Werkzeuge der empirischen Forschung}}\\
	\bigskip
	Abgabe: 24.06.2019\\
	Blatt 10\\[-1em]\\
	Pohl, Oliver \\	 577878\\  pohloliq\\[-1em]
}

\newcommand*{\QED}{\hfill\ensuremath{q.e.d.}}
\newcommand{\ex}[1]{\newpage\subsubsection*{Aufgabe #1.}}
\newcommand*{\simpleTable}[2]{
	\begin{tabular}{@{} #1 @{}}
		\toprule
		#2
		\bottomrule
	\end{tabular}
}
\DeclareMathOperator{\sig}{sig}

\newcommand\addvmargin[1]{
  \node[fit=(current bounding box),inner ysep=#1,inner xsep=0]{};
}

\begin{document}
	
	


	%%%%%%%%%%%%%%%%%%%% Aufgabe 29 %%%%%%%%%%%%%%%%%%%%%%%%%%%%%%%%%%%%%%%%%%%%%%%%%
\ex{29}

Daten:\\

\begin{tabular}{ | l | l | l | p{5cm} |}
	\hline
	Gruppe & FC & Rang  \\ \hline
	w & 23.01 & 1 \\ \hline
	w & 38.98 & 13  \\ \hline
	w & 29.65 & 8 \\ \hline
	w & 25.69 & 4 \\ \hline
	w & 37.17 & 10  \\ \hline
	w & 25.56 & 3 \\ \hline
	w & 29.37 & 7 \\ \hline
	w & 28.31 & 6  \\ \hline
	w & 33.60 & 9 \\ \hline
	w & 40.32 & 16 \\ \hline
	m & 43.41 & 18  \\ \hline
	m & 37.39 & 11 \\ \hline
	m & 65.11 & 21 \\ \hline
	m & 39.26 & 14  \\ \hline
	m & 48.79 & 20 \\ \hline
	m & 26.63 & 5 \\ \hline
	m & 43.76 & 19  \\ \hline
	m & 38.73 & 12 \\ \hline
	m & 41.94 & 17 \\ \hline
	m & 39.67 & 15  \\ \hline
	m & 23.85 & 2 \\ \hline
\end{tabular}
\\\\
Berechnung der Summe der Ränge für Stichprobe Gruppe "w":
\begin{align*}
S_1 
&= \sum_{j=1}^{10} R_{1j} \\
&= \sum_{j=1}^{10} R_{1j} = 1+13+8+4+10+3+7+6+9+16 \\
&= \sum_{j=1}^{10} R_{1j} = 77	\\
\end{align*}

Berechnung der Summe der Ränge für Stichprobe Gruppe "m":
\begin{align*}
S_2 
&= \sum_{j=1}^{11} R_{1j} \\
&= \sum_{j=1}^{11} R_{1j} = 18+11+21+14+20+5+19+12+17+15+2 \\
&= \sum_{j=1}^{11} R_{1j} = 154	\\
\end{align*}

Berechnung der Erwartungswerte(unter H0):

\begin{align*}
E(S_1)
&= \frac{10\cdot(10+11+1)}{2} \\
&=  110\\
\end{align*}

\begin{align*}
E(S_2)
&= \frac{11\cdot(11+10+1)}{2} \\
&=  121\\
\end{align*}

Berechnung der Varianzen:

\begin{align*}
Var(S_1)
&= \frac{10\cdot 11 \cdot (10+11+1)}{12} \\
&=  201.66 \\
\end{align*}


Berechnung der Teststatistik des Wilcoxon-Test:

\begin{align*}
Z
&= \frac{S-E(S)}{\sqrt{varS}} \\
&= \frac{77-110}{\sqrt{201.66}}  \\
&=  -2.32 \\
\end{align*}

Berechnung des p-Wertes:

\begin{align*}
p
&= 2 \cdot pnorm(-2.32) \\
&= 2 \cdot 0.0102  \\
&= 0.0204 \\
\end{align*}

Da der p-Wert = 0.0204 $<$ 0.05 lehnen wir die Nullhypothese ab.

	
	
\end{document}
