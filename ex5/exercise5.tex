\documentclass[a4paper,12pt]{article}
\usepackage[ngerman]{babel}
\usepackage[utf8]{inputenc}
\usepackage{tikz}
\usetikzlibrary{automata, arrows, fit}
\usepackage{marginnote}
\usepackage{lipsum}
\usepackage{graphicx}
\usepackage{epstopdf}
\usepackage{amsmath, tabu}
\usepackage{amsfonts}
\usepackage{geometry}
\usepackage{mathtools}
\geometry{
	headsep=0.5cm,
	headheight=2.5cm,
	marginparwidth=2cm,
	textheight=22cm,
}
\setlength{\headheight}{120pt}
\usepackage{fancyhdr}
\setlength\parindent{0pt}
\renewcommand{\familydefault}{\sfdefault}
\renewcommand{\arraystretch}{1.2}
\allowdisplaybreaks
\graphicspath{{../figures/}}


\pagestyle{fancy}
\fancyhf{}

\lhead{
	\textit{Humboldt-Universität zu Berlin, Institut der Informatik}\\
	\bigskip
	\textbf{\Large{Werkzeuge der empirischen Forschung}}\\
	\bigskip
	Abgabe: 20.05.2019\\
	Blatt 5\\[-1em]\\
	Pohl, Oliver \\	 577878\\  pohloliq\\[-1em]
}

\newcommand*{\QED}{\hfill\ensuremath{q.e.d.}}
\newcommand{\ex}[1]{\newpage\subsubsection*{Aufgabe #1.}}
\newcommand*{\simpleTable}[2]{
	\begin{tabular}{@{} #1 @{}}
		\toprule
		#2
		\bottomrule
	\end{tabular}
}
\DeclareMathOperator{\sig}{sig}

\newcommand\addvmargin[1]{
  \node[fit=(current bounding box),inner ysep=#1,inner xsep=0]{};
}

\begin{document}
	%%%%%%%%%%%%%%%%%%%% Aufgabe 11b1 %%%%%%%%%%%%%%%%%%%%%%%%%%%%%%%%%%%%%%%%%%%%%%%%%
	\ex{11b1}

	Berechnung der getrimmten und winsorisierten Mittel auf folgenden fiktiven Daten(1.5,2.7,2.8,3.0,3.1)\\\\
	
	
	
	
	\={X}_{win,1} = \frac{2 \cdot 2.7 + 2.8 + 2 \cdot 3.0}{5} = 2.84 \\\\
	
	\={X}_{trim,1} = \frac{2.7 + 2.8 + 3.0}{3} = 2.833
	
	
	

    
    
    
    	%%%%%%%%%%%%%%%%%%%% Aufgabe 11b2 %%%%%%%%%%%%%%%%%%%%%%%%%%%%%%%%%%%%%%%%%%%%%%%%%
    \ex{11b2}
    
    Berechnung der empirischen Streuung s^2:\\
    
    \={X} = 0.2 $\cdot$ (1.5 + 2.7 + 2.8 + 3.0 + 3.1)   = 2.62 \\
    
    \begin{align*}
    s^2  
    &=\frac{(1.5-2.62)^2 + (2.7-2.62)^2 + (2.8-2.62)^2 + (3.0-2.62)^2 + (3.1-2.62)^2}{4}\\
    &= \frac{1.254 + 0.0064 + 0.0324 + 0.1444 + 0.2304}{4}\\
    &= \frac{1.66}{4} = 0.415\\
    \end{align*}
    
    Berechnung des Interquartilrange IR :\\
    
    X_{0.25} = 2.7\\
    X_{0.75} = 3.0\\
    IR = X_{0.75} - X_{0.25} = 3.0 - 2.7 = 0.3\\\\
    
    Berechnung des MAD\\
    \begin{align*}
    med(|x_{i}-x_{0.5}|)
    &= med(|1.5-2.8|,|2.7-2.8|,|2.8-2.8|,|3.0-2.8|,|3.1-2.8|)\\
    &= med(|-1.3|,|-0.1|,|0|,|0.2|,|0.3|)\\
    &= med(1.3,0.1,0,0.2,0.3)\\
    &= 0.2\\
    \end{align*}
   
    Berechnung Ginis Mittelwertdifferenz:\\
    
   
    $\frac{1}{\binom{5}{2}}$ $\cdot$ \sum_{i<j}^{} |x_{i}-x_{j}|\\
   
    $\frac{1}{\binom{5}{2}}$ $\cdot$ ($|1.5-2.7|$+$|1.5-2.8|$+$|1.5-3.0|$+$|1.5-3.1|$+$|2.7-2.8|$+$|2.7-3.0|$+$|2.7-3.1|$
    +$|2.8-3.0|$+$|2.8-3.1|$+$|3.0-3.1|$)=\\
    $\frac{1}{\binom{5}{2}}$ $\cdot$ (1.2 + 1.3 + 1.5 + 1.6 + 0.1 + 0.3 + 0.4 + 0.2 + 0.3 + 0.1)=\\
     $\frac{1}{10}$ $\cdot$ (6.8) = 0.68
     \\\\
     \newpage
     Berechnung des S_{n}:\\
     
     S_{n}=1.1926 $\cdot$ med_{i}(med_{j}(|(x_{i}-x_{j})|)) \\\\
     
     i = 1\;  1.2 , 1.3 , 1.5 , 1.6  med(1.2 , 1.3 , 1.5 , 1.6) = 1.4\\
     i = 2\;  1.2 , 0.1 , 0.3 , 0.4  med(1.2 , 0.1 , 0.3 , 0.4) = 0.35\\
     i = 3\;  1.3 , 0.1 , 0.2 , 0.3  med(1.3 , 0.1 , 0.2 , 0.3) = 0.25\\
     i = 4\;  1.5 , 0.3 , 0.2 , 0.1  med(1.5 , 0.3 , 0.2 , 0.1) = 0.25\\
     i = 5\;  1.6 , 0.4 , 0.3 , 0.1  med(1.6 , 0.4 , 0.3 , 0.1) = 0.35\\\\
     
     med(1.4,0.35,0.25,0.25,0.35) = 0.35\\
     S_{n} = 1.1926 $\cdot$ 0.35 = 0.417 \\\\
     S_{n} (ohne Korrekturfaktor) = 0.35\\
     
     Berechnung des Q_{n}:\\
    
    Q_{n} = \{|x_{i}-x_{j}| \; , i<j\}
    
    i = 1\;  1.2 , 1.3 , 1.5 , 1.6\\
    i = 2\;  0.1 , 0.3 , 0.4 \\
    i = 3\;  0.2 , 0.3\\
    i = 4\;  0.1\\
    
    \{0.1 , 0.1 , 0.2 , 0.3 , 0.3 , 0.4 , 1.2 , 1.3 , 1.5 , 1.6\}
    
    h =  \left$\lfloor$ $\frac{5}{2}$ \right$\rfloor$ + 1 = 3 \\
    k = $\binom{3}{2}$ = 3\\\\
    
    Somit ist der Q_{n} = 0.2
    
	

\end{document}
