\documentclass[a4paper,12pt]{article}
\usepackage[ngerman]{babel}
\usepackage[utf8]{inputenc}
\usepackage{tikz}
\usetikzlibrary{automata, arrows, fit}
\usepackage{marginnote}
\usepackage{lipsum}
\usepackage{graphicx}
\usepackage{epstopdf}
\usepackage{amsmath, tabu}
\usepackage{amsfonts}
\usepackage{geometry}
\geometry{
	headsep=0.5cm,
	headheight=2.5cm,
	marginparwidth=2cm,
	textheight=22cm,
}
\setlength{\headheight}{120pt}
\usepackage{fancyhdr}
\setlength\parindent{0pt}
\renewcommand{\familydefault}{\sfdefault}
\renewcommand{\arraystretch}{1.2}
\allowdisplaybreaks
\graphicspath{{../figures/}}


\pagestyle{fancy}
\fancyhf{}

\lhead{
	\textit{Humboldt-Universität zu Berlin, Institut der Informatik}\\
	\bigskip
	\textbf{\Large{Werkzeuge der empirischen Forschung}}\\
	\bigskip
	Abgabe: 06.05.2019\\
	Blatt 3\\[-1em]\\
	Pohl, Oliver \\	 577878\\  pohloliq\\[-1em]
}

\newcommand*{\QED}{\hfill\ensuremath{q.e.d.}}
\newcommand{\ex}[1]{\newpage\subsubsection*{Aufgabe #1.}}
\newcommand*{\simpleTable}[2]{
	\begin{tabular}{@{} #1 @{}}
		\toprule
		#2
		\bottomrule
	\end{tabular}
}
\DeclareMathOperator{\sig}{sig}

\newcommand\addvmargin[1]{
  \node[fit=(current bounding box),inner ysep=#1,inner xsep=0]{};
}

\begin{document}
	%%%%%%%%%%%%%%%%%%%% Aufgabe 7a %%%%%%%%%%%%%%%%%%%%%%%%%%%%%%%%%%%%%%%%%%%%%%%%%
	\ex{5b}

	Berechnung der Varianz von  X~Bi(n,p)\\
	
	Zunächst einmal muss gezeigt werden, das E(X)=n*p:\\\\
	\begin{align*}
	E(X) = \sum_{k=1}^{n} k \cdot \binom{n}{k} \cdot p^{k} \cdot (1-p)^{n-k}
	&= \sum_{k=1}^{n} n \cdot \binom{n-1}{k-1} \cdot p^{k} \cdot (1-p)^{n-k}\\
	&= \sum_{k=1}^{n} n \cdot \binom{n-1}{k-1} \cdot p^{k-1} \cdot p \cdot (1-p)^{n-k}\\
	&= n \cdot p \cdot \sum_{k=0}^{n-1} \binom{n-1}{k} \cdot p^{k} \cdot (1-p)^{n-1-k}\\
	&= n \cdot p \cdot (p + (1-p))^{n-1}\\
	&= n \cdot p
	\end{align*}
	
	\begin{align*}
	Var(X)= E(X^{2})-E(X)^{2}
	&=E(X^{2})+E(X)-E(X)-E(X)^{2}\\
	&=E(X \cdot (X-1)) + E(X) - (E(X))^{2}\\
	&= \sum_{k=0}^{n} k \cdot (k-1) \cdot Bi(n,p) + n \cdot p - (n \cdot p)^{2}\\
	&= \sum_{k=2}^{n} n \cdot (n-1) \cdot \binom{n-2}{k-2} \cdot p^{k-2} \cdot p^{2} \cdot (1-p)^{n-k} + n \cdot p - (n \cdot p)^{2} \\
	&= p^{2} \cdot \sum_{k=0 }^{n-2} n \cdot (n-1) \cdot \binom{n-2}{k} \cdot p^{k} \cdot (1-p)^{n-2-k} n \cdot p - (n \cdot p)^{2} \\
	&= p^{2} \cdot n \cdot (n-1) \cdot \sum_{k=0}^{n-2} \binom{n-2}{k} \cdot p^{k} (1-p)^{n-2-k} n \cdot p - (n \cdot p)^{2}\\
	&= p^{2} \cdot n \cdot(n-1) + n \cdot p - (n \cdot p)^{2}\\
	&= p^{2} \cdot n^{2} - p^{2} \cdot n + n \cdot p - n \\
	&= n \cdot (1-p) \cdot p
	\end{align*}
	

\end{document}
