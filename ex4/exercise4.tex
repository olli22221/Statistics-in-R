\documentclass[a4paper,12pt]{article}
\usepackage[ngerman]{babel}
\usepackage[utf8]{inputenc}
\usepackage{tikz}
\usetikzlibrary{automata, arrows, fit}
\usepackage{marginnote}
\usepackage{lipsum}
\usepackage{graphicx}
\usepackage{epstopdf}
\usepackage{amsmath, tabu}
\usepackage{amsfonts}
\usepackage{geometry}
\geometry{
	headsep=0.5cm,
	headheight=2.5cm,
	marginparwidth=2cm,
	textheight=22cm,
}
\setlength{\headheight}{120pt}
\usepackage{fancyhdr}
\setlength\parindent{0pt}
\renewcommand{\familydefault}{\sfdefault}
\renewcommand{\arraystretch}{1.2}
\allowdisplaybreaks
\graphicspath{{../figures/}}


\pagestyle{fancy}
\fancyhf{}

\lhead{
	\textit{Humboldt-Universität zu Berlin, Institut der Informatik}\\
	\bigskip
	\textbf{\Large{Werkzeuge der empirischen Forschung}}\\
	\bigskip
	Abgabe: 13.05.2019\\
	Blatt 4\\[-1em]\\
	Pohl, Oliver \\	 577878\\  pohloliq\\[-1em]
}

\newcommand*{\QED}{\hfill\ensuremath{q.e.d.}}
\newcommand{\ex}[1]{\newpage\subsubsection*{Aufgabe #1.}}
\newcommand*{\simpleTable}[2]{
	\begin{tabular}{@{} #1 @{}}
		\toprule
		#2
		\bottomrule
	\end{tabular}
}
\DeclareMathOperator{\sig}{sig}

\newcommand\addvmargin[1]{
  \node[fit=(current bounding box),inner ysep=#1,inner xsep=0]{};
}

\begin{document}
	%%%%%%%%%%%%%%%%%%%% Aufgabe 10a %%%%%%%%%%%%%%%%%%%%%%%%%%%%%%%%%%%%%%%%%%%%%%%%%
	\ex{10a}

	Herleitung der Maximum-Likelihood-Schätzung für $\lambda$ \\\\
	
	
	\begin{align*}
	L(\lambda) = \prod_{i=1}^{n} \exp^{-\lambda \cdot \frac{\lambda^{X_i}}{X_i!}}
	\Leftrightarrow ln(L(\lambda)) = &\sum_{i=1}^{n} ln(\exp^{-\lambda \cdot \frac{\lambda^{X_i}}{X_i!}})\\
	&= \sum_{i=1}^{n} (-\lambda + X_i \cdot ln(\lambda) - ln(X_i!))\\
	&= -n \cdot \lambda + ln(\lambda) \cdot \sum_{i=1}^{n} X_i - \sum_{i=1}^{n} ln(X_i!) \Leftrightarrow ln(L(\lambda)) \frac{d}{d\lambda} = -n + \frac{1}{\lambda} \cdot \sum_{i=1}^{n} X_1\\
	& = -n + \frac{1}{\lambda} \cdot \sum_{i=1}^{n} X_1 = 0 \Rightarrow \frac{1}{n} \cdot \sum_{1}^{n} X_i = \lambda_{maximumlikelihood}
	\end{align*}
	
	Herleitung der Momenten-Schätzung für $\lambda$ \\\\
	\begin{align*}
	Seien X_1, ... , X_n ~ Poi(\lambda)\\
	\lambda = E(X_i) \Rightarrow \lambda = \stackrel{-}{X}
    \end{align*}
	

\end{document}
