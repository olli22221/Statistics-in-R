\documentclass[a4paper,12pt]{article}
\usepackage[ngerman]{babel}
\usepackage[utf8]{inputenc}
\usepackage{tikz}
\usetikzlibrary{automata, arrows, fit}
\usepackage{marginnote}
\usepackage{lipsum}
\usepackage{graphicx}
\usepackage{epstopdf}
\usepackage{amsmath, tabu}
\usepackage{amsfonts}
\usepackage{geometry}
\usepackage{mathtools}
\geometry{
	headsep=0.5cm,
	headheight=2.5cm,
	marginparwidth=2cm,
	textheight=22cm,
}
\setlength{\headheight}{120pt}
\usepackage{fancyhdr}
\setlength\parindent{0pt}
\renewcommand{\familydefault}{\sfdefault}
\renewcommand{\arraystretch}{1.2}
\allowdisplaybreaks
\graphicspath{{../figures/}}


\pagestyle{fancy}
\fancyhf{}

\lhead{
	\textit{Humboldt-Universität zu Berlin, Institut der Informatik}\\
	\bigskip
	\textbf{\Large{Werkzeuge der empirischen Forschung}}\\
	\bigskip
	Abgabe: 27.05.2019\\
	Blatt 6\\[-1em]\\
	Pohl, Oliver \\	 577878\\  pohloliq\\[-1em]
}

\newcommand*{\QED}{\hfill\ensuremath{q.e.d.}}
\newcommand{\ex}[1]{\newpage\subsubsection*{Aufgabe #1.}}
\newcommand*{\simpleTable}[2]{
	\begin{tabular}{@{} #1 @{}}
		\toprule
		#2
		\bottomrule
	\end{tabular}
}
\DeclareMathOperator{\sig}{sig}

\newcommand\addvmargin[1]{
  \node[fit=(current bounding box),inner ysep=#1,inner xsep=0]{};
}

\begin{document}
	%%%%%%%%%%%%%%%%%%%% Aufgabe 14a %%%%%%%%%%%%%%%%%%%%%%%%%%%%%%%%%%%%%%%%%%%%%%%%%
	\ex{14a}
	
	Berechnung des Pearson-Korrelationskoeffizienten:\\\\
	
	für $X_{n}$ = k=5 \\
	
	\={X} = $\frac{1}{5}$ $\cdot$ (1 + 2 + 3 + 4 + 5) = 3\\
	\={Y} = $\frac{1}{5}$ $\cdot$ (1 + 2 + 3 + 4 + 5) = 3\\
	
	
	\begin{align*}
	S_{xy} 
	&= \frac{1}{4} \cdot \sum_{i=1}^{5} (x_{i}-3) \cdot (y_{i}-3) \\
	&= \frac{1}{4} \cdot( (1-3)\cdot (1-3) + (2-3)\cdot (2-3) \\
	&+ (3-3)\cdot (3-3) + (4-3)\cdot (4-3) + (5-3)\cdot (5-3))\\
	&=  \frac{1}{4} \cdot( 4 + 1 + 0 + 1 + 4  )	\\
	&=  \frac{1}{4} \cdot 10 \\
	&= \frac{10}{4}\\\\
	\end{align*}
	\begin{align*}
	S_{x}^2 
	&= \frac{1}{4} \cdot \sum_{i=1}^{5} (x_{i}-3) \cdot (x_{i}-3) \\
	&= \frac{1}{4} \cdot( (1-3)\cdot (1-3) + (2-3)\cdot (2-3) \\
	&+ (3-3)\cdot (3-3) + (4-3)\cdot (4-3) + (5-3)\cdot (5-3))\\
	&=  \frac{1}{4} \cdot( 4 + 1 + 0 + 1 + 4  )	\\
	&=  \frac{1}{4} \cdot 10 \\
	&= \frac{10}{4}\\
	S_{x} =\frac{\sqrt{10}}{2}\\\\
	\end{align*}
	\newpage
	\begin{align*}
	S_{y}^2 
	&= \frac{1}{4} \cdot \sum_{i=1}^{5} (y_{i}-3) \cdot (y_{i}-3) \\
	&= \frac{1}{4} \cdot( (1-3)\cdot (1-3) + (2-3)\cdot (2-3) \\
	&+ (3-3)\cdot (3-3) + (4-3)\cdot (4-3) + (5-3)\cdot (5-3))\\
	&=  \frac{1}{4} \cdot( 4 + 1 + 0 + 1 + 4  )	\\
	&=  \frac{1}{4} \cdot 10 \\
	&= \frac{10}{4}\\
	S_{y} =\frac{\sqrt{10}}{2}\\
	\end{align*}
	
	\begin{align*}
	r_{xy} 
	&= \frac{S_{xy}}{S_{x} \cdot S_{y}}\\
	&= \frac{\frac{10}{4}}{\frac{\sqrt{10}}{2} \cdot \frac{\sqrt{10}}{2}}\\
	&= \frac{\frac{10}{4}}{\frac{10}{4}}\\
	&= 1\\
	\end{align*}
	
	für $X_{n}$ = k=10 \\
	\={X} = $\frac{1}{5}$ $\cdot$ (1 + 2 + 3 + 4 + 10) = 4\\
	$\={Y}$ = $\frac{1}{5}$ $\cdot$ (1 + 2 + 3 + 4 + 5) = 3\\
	\newpage	
	\begin{align*}
	S_{xy} 
	&= \frac{1}{4} \cdot \sum_{i=1}^{5} (x_{i}-4) \cdot (y_{i}-3) \\
	&= \frac{1}{4} \cdot( (1-4)\cdot (1-3) + (2-4)\cdot (2-3) \\
	&+ (3-4) \cdot (3-3) + (4-4)\cdot (4-3) + (10-4)\cdot (5-3))\\
	&=  \frac{1}{4} \cdot( 6 + 2 + 0 + 0 + 12  )	\\
	&=  \frac{1}{4} \cdot 20 \\
	&= \frac{20}{4}\\
	&= 5\\
	\end{align*}

	\begin{align*}
	S_{x}^2 
	&= \frac{1}{4} \cdot \sum_{i=1}^{5} (x_{i}-3) \cdot (x_{i}-3) \\
	&= \frac{1}{4} \cdot( (1-4)\cdot (1-4) + (2-4)\cdot (2-4)\\
	& + (3-4)\cdot (3-4) + (4-4)\cdot (4-4) + (10-4)\cdot (10-4))\\
	&=  \frac{1}{4} \cdot( 9 + 4 + 1 + 0 + 36  )	\\
	&=  \frac{1}{4} \cdot 50 \\
	&= \frac{50}{4}\\
	S_{x} =\frac{\sqrt{50}}{2}\\
	\end{align*}
	\newpage
	\begin{align*}
	S_{y}^2 
	&= \frac{1}{4} \cdot \sum_{i=1}^{5} (y_{i}-3) \cdot (y_{i}-3) \\
	&= \frac{1}{4} \cdot( (1-3)\cdot (1-3) + (2-3)\cdot (2-3) \\
	&+ (3-3)\cdot (3-3) + (4-3)\cdot (4-3) + (5-3)\cdot (5-3))\\
	&=  \frac{1}{4} \cdot( 4 + 1 + 0 + 1 + 4  )	\\
	&=  \frac{1}{4} \cdot 10 \\
	&= \frac{10}{4}\\
	S_{y} =\frac{\sqrt{10}}{2}\\
	\end{align*}
	
	\begin{align*}
	r_{xy} 
	&= \frac{S_{xy}}{S_{x} \cdot S_{y}}\\
	&= \frac{5}{\frac{\sqrt{50}}{2} \cdot \frac{\sqrt{10}}{2}}\\
	&= 0.89\\
	\end{align*}
	\\\\
	für $X_{n}$ = k=100 \\
	\={X} = $\frac{1}{5}$ $\cdot$ (1 + 2 + 3 + 4 + 100) = 22\\
	$\={Y}$ = $\frac{1}{5}$ $\cdot$ (1 + 2 + 3 + 4 + 5) = 3\\
	\newpage	
	\begin{align*}
	S_{xy} 
	&= \frac{1}{4} \cdot \sum_{i=1}^{5} (x_{i}-22) \cdot (y_{i}-3) \\
	&= \frac{1}{4} \cdot( (1-22)\cdot (1-3) + (2-22)\cdot (2-3) \\
	&+ (3-22) \cdot (3-3) + (4-22)\cdot (4-3) + (100-22)\cdot (5-3))\\
	&=  \frac{1}{4} \cdot( 42 + 20 + 0 - 18 + 156  )	\\
	&=  \frac{1}{4} \cdot 200 \\
	&= \frac{200}{4}\\
	&= 50\\
	\end{align*}
	
	\begin{align*}
	S_{x}^2 
	&= \frac{1}{4} \cdot \sum_{i=1}^{5} (x_{i}-22) \cdot (x_{i}-22) \\
	&= \frac{1}{4} \cdot( (1-22)\cdot (1-22) + (2-22)\cdot (2-22) \\
	&+ (3-22)\cdot (3-22) + (4-22)\cdot (4-22) + (100-22)\cdot (100-22))\\
	&=  \frac{1}{4} \cdot( 441 + 400 + 361 + 324 + 6084  )	\\
	&=  \frac{1}{4} \cdot 7610 \\
	S_{x} =43.62\\
	\end{align*}
	\newpage
	\begin{align*}
	S_{y}^2 
	&= \frac{1}{4} \cdot \sum_{i=1}^{5} (y_{i}-3) \cdot (y_{i}-3) \\
	&= \frac{1}{4} \cdot( (1-3)\cdot (1-3) + (2-3)\cdot (2-3) \\
	&+ (3-3)\cdot (3-3) + (4-3)\cdot (4-3) + (5-3)\cdot (5-3))\\
	&=  \frac{1}{4} \cdot( 4 + 1 + 0 + 1 + 4  )	\\
	&=  \frac{1}{4} \cdot 10 \\
	&= \frac{10}{4}\\
	S_{y} =\frac{\sqrt{10}}{2}\\
	\end{align*}
	
	\begin{align*}
	r_{xy} 
	&= \frac{S_{xy}}{S_{x} \cdot S_{y}}\\
	&= \frac{59}{43.62 \cdot \frac{\sqrt{10}}{2}}\\
	&= \frac{50}{69} \\
	&= 0.72
	\end{align*}
	\\\\
	\newpage
	für $X_{n}$ = k=1000 \\
	\={X} = $\frac{1}{5}$ $\cdot$ (1 + 2 + 3 + 4 + 1000) = 202\\
	$\={Y}$ = $\frac{1}{5}$ $\cdot$ (1 + 2 + 3 + 4 + 5) = 3\\
	
	\begin{align*}
	S_{xy} 
	&= \frac{1}{4} \cdot \sum_{i=1}^{5} (x_{i}-202) \cdot (y_{i}-3) \\
	&= \frac{1}{4} \cdot( (1-202)\cdot (1-3) + (2-202)\cdot (2-3) \\
	&+ (3-202) \cdot (3-3) + (4-202)\cdot (4-3) + (1000-202)\cdot (5-3))\\
	&=  \frac{1}{4} \cdot( 402 + 200 + 0 - 198 + 1596  )	\\
	&=  \frac{1}{4} \cdot 200 \\
	&= \frac{2000}{4}\\
	&= 500\\
	\end{align*}
	
	\begin{align*}
	S_{x}^2 
	&= \frac{1}{4} \cdot \sum_{i=1}^{5} (x_{i}-202) \cdot (x_{i}-202) \\
	&= \frac{1}{4} \cdot( (1-202)\cdot (1-202) + (2-202)\cdot (2-202)\\
	& + (3-202)\cdot (3-202) + (4-202)\cdot (4-202) + (1000-202)\cdot (1000-202))\\
	&=  \frac{1}{4} \cdot( 40401 + 40000 + 39601 + 39204 + 636804  )	\\
	&=  \frac{1}{4} \cdot 796010 \\
	S_{x} = 446\\
	\end{align*}
	\newpage
	\begin{align*}
	S_{y}^2 
	&= \frac{1}{4} \cdot \sum_{i=1}^{5} (y_{i}-3) \cdot (y_{i}-3) \\
	&= \frac{1}{4} \cdot( (1-3)\cdot (1-3) + (2-3)\cdot (2-3) \\
	&+ (3-3)\cdot (3-3) + (4-3)\cdot (4-3) + (5-3)\cdot (5-3))\\
	&=  \frac{1}{4} \cdot( 4 + 1 + 0 + 1 + 4  )	\\
	&=  \frac{1}{4} \cdot 10 \\
	&= \frac{10}{4}\\
	S_{y} =\frac{\sqrt{10}}{2}\\
	\end{align*}
	
	\begin{align*}
	r_{xy} 
	&= \frac{S_{xy}}{S_{x} \cdot S_{y}}\\
	&= \frac{500}{446 \cdot \frac{\sqrt{10}}{2}}\\
	&= \frac{500}{705} \\
	&= 0.709
	\end{align*}
	\\\\
	
	\newpage
	
	Berechnung des Spearman-Rangkorrelationskoeffizienten:\\\\
	
	für $X_{n}$ = k=5,10,100,1000 \\
	
	\begin{align*}
	r_{s} 
	&= 1 - \frac{6*\sum_{i=1}^{5}(R_{i}-S_{i})^2}{5 \cdot (5^2 - 1)} \\
	&= 1 - \frac{6*\sum_{i=1}^{5}(R_{i}-S_{i})^2}{120} \\
	&= 1 - \frac{6*((1-1)^2 + (2-2)^2 + (3-3)^2 + (4-4)^2 + (5-5)^2)}{120}	\\
	&= 1 - \frac{6*(0)}{120}  \\
	&= 1 - \frac{0}{120}\\
	&= 1 - 0\\
	&= 1\\
	\end{align*}
	
	Berechnung des Kendall-Rangkorrelationskoeffizienten:\\\\
	
	für $X_{n}$ = k=5 \\
	
	\begin{align*}
	\tau 
	&= \frac{1}{\binom{5}{2}} \cdot \sum_{i<j}^{} a_{ij} \\
	&= \frac{1}{\binom{5}{2}} \cdot sgn((1-2)*(1-2)) + sgn((1-3)*(1-3))\\
	& +sgn((1-4)*(1-4)) +sgn((1-5)*(1-5)) +sgn((2-3)*(2-3)) \\
	&+ sgn((2-4)*(2-4)) + sgn((2-5)*(2-5)) + sgn((3-4)*(3-4))\\
	& + sgn((3-5)*(3-5)) + sgn((4-5)*(4-5))  \\
	&= \frac{1}{\binom{5}{2}} \cdot 1 + 1 + 1 + 1 + 1 + 1 + 1 +	1 + 1 + 1 \\
	&= \frac{1}{\binom{5}{2}} \cdot 10  \\
	&= \frac{10}{10} \\
	&= 1 
	\end{align*}
	
	\newpage
	
	für $X_{n}$ = k=10 \\
	
	\begin{align*}
	\tau 
	&= \frac{1}{\binom{5}{2}} \cdot \sum_{i<j}^{} a_{ij} \\
	&= \frac{1}{\binom{5}{2}} \cdot sgn((1-2)*(1-2)) \\
	&+ sgn((1-3)*(1-3)) +sgn((1-4)*(1-4)) +sgn((1-10)*(1-5))\\
	& +sgn((2-3)*(2-3)) + sgn((2-4)*(2-4)) + sgn((2-10)*(2-5)) \\
	&+ sgn((3-4)*(3-4)) + sgn((3-10)*(3-5)) + sgn((4-10)*(4-5))  \\
	&= \frac{1}{\binom{5}{2}} \cdot 1 + 1 + 1 + 1 + 1 + 1 + 1 +	1 + 1 + 1 \\
	&= \frac{1}{\binom{5}{2}} \cdot 10  \\
	&= \frac{10}{10} \\
	&= 1 
	\end{align*}
	
		für $X_{n}$ = k=100 \\
	
	\begin{align*}
	\tau 
	&= \frac{1}{\binom{5}{2}} \cdot \sum_{i<j}^{} a_{ij} \\
	&= \frac{1}{\binom{5}{2}} \cdot sgn((1-2)*(1-2)) \\
	&+ sgn((1-3)*(1-3)) +sgn((1-4)*(1-4)) +sgn((1-100)*(1-5)) \\
	&+sgn((2-3)*(2-3)) + sgn((2-4)*(2-4)) + sgn((2-100)*(2-5)) \\
	&+ sgn((3-4)*(3-4)) + sgn((3-100)*(3-5)) + sgn((4-100)*(4-5))  \\
	&= \frac{1}{\binom{5}{2}} \cdot 1 + 1 + 1 + 1 + 1 + 1 + 1 +	1 + 1 + 1 \\
	&= \frac{1}{\binom{5}{2}} \cdot 10  \\
	&= \frac{10}{10} \\
	&= 1 
	\end{align*}
	\newpage
	
	für $X_{n}$ = k=1000 \\
	
	\begin{align*}
	\tau 
	&= \frac{1}{\binom{5}{2}} \cdot \sum_{i<j}^{} a_{ij} \\
	&= \frac{1}{\binom{5}{2}} \cdot sgn((1-2)*(1-2)) + sgn((1-3)*(1-3)) +sgn((1-4)*(1-4))\\  &+sgn((1-1000)*(1-5)) +sgn((2-3)*(2-3))\\
	& + sgn((2-4)*(2-4)) + sgn((2-1000)*(2-5))\\
	&+sgn((3-4)*(3-4)) + sgn((3-1000)*(3-5)) + sgn((4-1000)*(4-5))  \\
	&= \frac{1}{\binom{5}{2}} \cdot 1 + 1 + 1 + 1 + 1 + 1 + 1 +	1 + 1 + 1 \\
	&= \frac{1}{\binom{5}{2}} \cdot 10  \\
	&= \frac{10}{10} \\
	&= 1 
	\end{align*}
	
	
		%%%%%%%%%%%%%%%%%%%% Aufgabe 14a %%%%%%%%%%%%%%%%%%%%%%%%%%%%%%%%%%%%%%%%%%%%%%%%%
	\ex{14d}
	
	Grenzwert des Kendall-Korrelationskoeffizient:
	
		$\lim$ $\tau$ = \pm 1  \\
		$-1 \leq \tau  \leq $1
	
	
	
	\end{document}
