\documentclass[a4paper,12pt]{article}
\usepackage[ngerman]{babel}
\usepackage[utf8]{inputenc}
\usepackage{tikz}
\usetikzlibrary{automata, arrows, fit}
\usepackage{marginnote}
\usepackage{lipsum}
\usepackage{graphicx}
\usepackage{epstopdf}
\usepackage{amsmath, tabu}
\usepackage{amsfonts}
\usepackage{geometry}
\geometry{
	headsep=0.5cm,
	headheight=2.5cm,
	marginparwidth=2cm,
	textheight=22cm,
}
\setlength{\headheight}{120pt}
\usepackage{fancyhdr}
\setlength\parindent{0pt}
\renewcommand{\familydefault}{\sfdefault}
\renewcommand{\arraystretch}{1.2}
\allowdisplaybreaks
\graphicspath{{../figures/}}


\pagestyle{fancy}
\fancyhf{}

\lhead{
	\textit{Humboldt-Universität zu Berlin, Institut der Informatik}\\
	\bigskip
	\textbf{\Large{Werkzeuge der empirischen Forschung}}\\
	\bigskip
	Abgabe: 29.04.2019\\
	Blatt 2\\[-1em]\\
	Pohl, Oliver \\	 577878\\  pohloliq\\[-1em]
}

\newcommand*{\QED}{\hfill\ensuremath{q.e.d.}}
\newcommand{\ex}[1]{\newpage\subsubsection*{Aufgabe #1.}}
\newcommand*{\simpleTable}[2]{
	\begin{tabular}{@{} #1 @{}}
		\toprule
		#2
		\bottomrule
	\end{tabular}
}
\DeclareMathOperator{\sig}{sig}

\newcommand\addvmargin[1]{
  \node[fit=(current bounding box),inner ysep=#1,inner xsep=0]{};
}

\begin{document}
	%%%%%%%%%%%%%%%%%%%% Aufgabe 5b %%%%%%%%%%%%%%%%%%%%%%%%%%%%%%%%%%%%%%%%%%%%%%%%%
	\ex{5b}

	P($X > 6$), wenn  X~Bi(10,5/6)\\
	\begin{align*}
	P(X > 6) &=1- P(X <= 6)\\
			 &= 1- \sum_{i=0}^{6} \binom{10}{i} \cdot \frac{5}{6}^{i} \cdot (1-5/6)^{10-i}\\
			 &= 1- 0.06972784\\
			 &= 0.9302722\\
	\end{align*}
	
	P($X > 6$), wenn  X~Poi(50/6)\\
	\begin{align*}
	P(X > 6) &=1- P(X <= 6)\\
	&= 1- \sum_{i=0}^{6} \frac{\frac{50}{6}^{i}}{i!} \cdot \exp^{-\frac{50}{6}} \\
	&= 1- 0.2743767 \\
	&= 0.7256233\\
	\end{align*}
	
	P($X > 6$), wenn  X~Geo(5/6)\\
	\begin{align*}
	P(X > 6) &=1- P(X <= 6)\\
	&= 1- \sum_{i=1}^{6} \frac{5}{6} \cdot (1-\frac{5}{6})^{i-1}\\ 
	&= 1- 0.9999964 \\
	&= 0.00000359999\\
	\end{align*}
	
	P($X > 6$), wenn  X~N(6,9)\\
	\begin{align*}
	P(X > 6) &=1- P(X <= 6)\\
	&= 1- \sum_{i=- \inf}^{6} \frac{1}{\sqrt{2 \cdot \pi \cdot 9}} \cdot \exp^{-\frac{(i-6)^{2}}{2 \cdot 9}}  \\
	&= 1- 0.5 \\
	&= 0.5\\
	\end{align*}
	

\end{document}
